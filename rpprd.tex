%%%%%%%%%%%%%%%%%%%%%%%%%%%%%%%%%%%%%%%%%%%%%%%%%%%%%%%%%%%%%%%%%%%%%%%%%%%%%%%
% Set a class and general configuration
\documentclass[a4paper,onecolumn,10pt]{article}

%%%%%%%%%%%%%%%%%%%%%%%%%%%%%%%%%%%%%%%%%%%%%%%%%%%%%%%%%%%%%%%%%%%%%%%%%%%%%%%
% Set variables with the title, authors, etc.
\newcommand{\Titulo}{Relatório de Participação no Programa de Recepção Docente}
\newcommand{\Nome}{Leonardo Uieda}
\newcommand{\Cargo}{Professor Doutor}
\newcommand{\Email}{uieda@usp.br}

%%%%%%%%%%%%%%%%%%%%%%%%%%%%%%%%%%%%%%%%%%%%%%%%%%%%%%%%%%%%%%%%%%%%%%%%%%%%%%%
% Import the required packages
\usepackage[utf8]{inputenc}
\usepackage[TU]{fontenc}
\usepackage[brazil]{babel}
%\usepackage[english]{babel}
\usepackage{graphicx}
\usepackage{hyperref}
\usepackage{fancyhdr}
\usepackage{geometry}
\usepackage{microtype}
\usepackage{xcolor}
% improved urls with proper hyphenation
\usepackage{xurl}
% Use a different font
\usepackage[default,scale=0.95]{opensans}
% Icons and fonts (requires using xelatex or luatex)
\usepackage{fontawesome5}
\usepackage{academicons}
% Control the font size
\usepackage{anyfontsize}
\usepackage{setspace}
% Generate random text
\usepackage{lipsum}
% Better left and right align
\usepackage{ragged2e}

%%%%%%%%%%%%%%%%%%%%%%%%%%%%%%%%%%%%%%%%%%%%%%%%%%%%%%%%%%%%%%%%%%%%%%%%%%%%%%%
% Configuration of the document

\geometry{%
  left=25mm,
  right=15mm,
  top=15mm,
  bottom=15mm,
  headsep=15mm,
  headheight=15mm,
  footskip=7mm,
  includehead=true,
  includefoot=true
}

% Control line spacing
\onehalfspacing
\newcommand{\Padding}{\vspace{0.5cm}}

% Custom colors
\definecolor{darkgray}{gray}{0.4}
\definecolor{mediumgray}{gray}{0.5}
\definecolor{lightgray}{gray}{0.9}
\definecolor{mediumblue}{HTML}{2060c2}
\definecolor{lightblue}{HTML}{f7faff}

% Make urls use the same font as every other text
\urlstyle{same}

% Configure hyperref and add PDF metadata
\hypersetup{
    colorlinks,
    allcolors=mediumblue,
    pdftitle={\Titulo},
    pdfauthor={\Nome},
    breaklinks=true,
}

% Configure header and footer
\newcommand{\HeaderFont}{\footnotesize\color{mediumgray}}
\pagestyle{fancy}
\fancyhf{}
\rhead{%
  \includegraphics[height=1.5cm]{figures/usp.png}
}
\lhead{%
  \includegraphics[height=1.5cm]{figures/iag.png}
}
\cfoot{%
  \HeaderFont{}
  \Centering
  \onehalfspacing
  Rua do Matão, 1226 - São Paulo, SP, Brasil, 05508-090
  \newline
  E-mail: \href{mailto:\Email}{\Email};
  Website: \href{https://www.iag.usp.br}{www.iag.usp.br}
}
\renewcommand{\headrulewidth}{0pt}
\renewcommand{\footrulewidth}{1pt}
\preto{\footrule}{\color{lightgray}}

%%%%%%%%%%%%%%%%%%%%%%%%%%%%%%%%%%%%%%%%%%%%%%%%%%%%%%%%%%%%%%%%%%%%%%%%%%%%%%%
\begin{document}

\noindent\textbf{\large \Titulo}
\Padding

No dia 29 de maio de 2025, participei o evento obrigatório ``Boas-vindas,
docentes'' organizado pela CERT no Anfiteatro Camargo Guarnieri da Cidade
Universitária em São Paulo -- SP junto com 324 outros novos docentes.
O evento foi iniciado pela apresentação ``Movimento Cultural'' da Orquestra
Sinfônica da Universidade de São Paulo.
Em seguida, tivemos cerca de duas horas e trinta minutos de palestras diversas.
Primeiro tivemos uma abertura do Reitor Carlos Gilberto Carlotti Junior,
seguido de uma palestra do Assessor da Diretoria Científica da FAPESP Sylvio
Roberto Accioly Canuto.
A próxima apresentação foi feita pela Presidente da CERT Anna Helena Reali
Costa.
Em seguida, o ``Painel de Diálogo com Inovas'' contou com apresentações do
diretor do Centro de Inovação InovaUSP, Marcelo Knörich Zuffo, e o coordenador
do polo do InovaUSP em São Carlos, Tito José Bonagamba.
O painel foi moderado por Luiz Henrique Catalani, Coordenador da Agência
USP de Inovação.
A próxima apresentação foi intitulada ``USP em Números'' e feita por Fátima de
Lourdes dos Santos Nunes, Coordenadora do Escritório de gestão de Indicadores
de Desempenho Acadêmico.
Por fim, tivemos uma pausa no evento para almoço individual.

Na parte da tarde, a programação contou com mais uma sessão de aproximadamente
duas horas de palestras, iniciando com uma apresentação do vídeo
institucional da USP e finalizando com
``Painel de Diálogo'' com os pró-reitores Aluísio Segurado (Graduação), Rodrigo
Calado (Pós-Graduação), Paulo Nussenzveig (Pesquisa e Inovação), Ana Lanna
(Inclusão e Pertencimento) e Hussam El Dine Zaher (adjunto de Cultura
e Extensão Universitária).
A mediação do painel foi feita pelo chefe do Gabinete do Reitor,
Arlindo Philippi Junior.
Ao final, tivemos uma breve sessão de perguntas e o encerramento da reunião.


\Padding
\noindent\Nome{}
\\[0.25cm]
{
\color{mediumgray}
\small
\Cargo
\\
Departamento de Geofísica
\\
Instituto de Astronomia, Geofísica e Ciências Atmosféricas
\\
Universidade de São Paulo
}
\end{document}
