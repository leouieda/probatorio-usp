% Project de atuação no estágio probatório - 2023-2025
%
%%%%%%%%%%%%%%%%%%%%%%%%%%%%%%%%%%%%%%%%%%%%%%%%%%%%%%%%%%%%%%%%%%%%%%%%%%%%%%%
% Set a class and general configuration
\documentclass[12pt,a4paper,oneside]{book}

%%%%%%%%%%%%%%%%%%%%%%%%%%%%%%%%%%%%%%%%%%%%%%%%%%%%%%%%%%%%%%%%%%%%%%%%%%%%%%%
% Set variables with the title, authors, etc.
\newcommand{\Title}{Projeto de Estágio Docente Final}
\newcommand{\Year}{2025}
\newcommand{\Date}{Agosto de \Year{}}
\newcommand{\Author}{Leonardo Uieda}

%%%%%%%%%%%%%%%%%%%%%%%%%%%%%%%%%%%%%%%%%%%%%%%%%%%%%%%%%%%%%%%%%%%%%%%%%%%%%%%
% Import the required packages
\usepackage[utf8]{inputenc}
\usepackage[TU]{fontenc}
\usepackage[brazil]{babel}
\usepackage{amsmath}
\usepackage{amssymb}
\usepackage{graphicx}
\usepackage{hyperref}
\usepackage{fancyhdr}
\usepackage{geometry}
\usepackage{booktabs}
\usepackage{microtype}
\usepackage{siunitx}
\usepackage{xcolor}
% Improved urls with proper hyphenation
\usepackage{xurl}
% Tweak the look of captions
\usepackage{caption}
% To control the style of section titles
\usepackage{titlesec}
% Import natbib and doi packages
\usepackage[round,authoryear,sort]{natbib}
% Add the bibliography to the table of contents
\usepackage[nottoc,chapter]{tocbibind}
% Reference sections by name
\usepackage{nameref}
% Better handling of footnotes inside summary boxes
\usepackage{footmisc}
% Show dois as links on references
\usepackage{doi}
% Remove extra space between references
\usepackage{natbibspacing}
% Use a different font
\usepackage[scaled=0.9,sfdefault]{notomath}
% Icons and fonts (requires using xelatex or luatex)
\usepackage{fontawesome5}
\usepackage{academicons}
% Control the font size
\usepackage{anyfontsize}
\usepackage{setspace}
% To get the number of pages in the document
\usepackage{lastpage}
\usepackage{ragged2e}
% Control over enumerate and itemize
\usepackage{enumitem}
% To define custom environments
\usepackage{environ}
\usepackage{mdframed}
% To control hyphenation for individual blocks of text
\usepackage{hyphenat}
\usepackage{lipsum}

%%%%%%%%%%%%%%%%%%%%%%%%%%%%%%%%%%%%%%%%%%%%%%%%%%%%%%%%%%%%%%%%%%%%%%%%%%%%%%%
% Configuration of the document

\geometry{%
  left=25mm,
  right=25mm,
  top=20mm,
  bottom=15mm,
  headsep=0mm,
  headheight=0mm,
  footskip=5mm,
  includehead=true,
  includefoot=true
}

% Control line and table row spacing
\onehalfspacing
\renewcommand{\arraystretch}{1.5}

% Make urls use the same font as every other text
\urlstyle{same}

% Set the spacing between bibliography entries (requires natbib)
\setlength{\bibsep}{0pt}

% Prevent footnotes from being broken into multiple pages
\interfootnotelinepenalty=10000

% Customize how Chapter headings are displayed
\titleclass{\chapter}{straight}
\titleformat{\chapter}[display]{\normalfont}{}{0pt}{\onehalfspacing\ifnum\thechapter>0 \Large\thechapter. \fi\huge}[\titlerule]
\titlespacing*{\chapter}{0pt}{10pt}{20pt}

% Custom colors
\definecolor{darkgray}{gray}{0.4}
\definecolor{mediumgray}{gray}{0.5}
\definecolor{lightgray}{gray}{0.9}
\definecolor{mediumblue}{HTML}{2060c2}
\definecolor{lightblue}{HTML}{f7faff}

% Configure captions
\captionsetup[table]{position=below,skip=0pt}
\captionsetup{labelfont=bf,font={small,color=darkgray},skip=10pt}

% Configure hyperref and add PDF metadata
\hypersetup{
    colorlinks,
    allcolors=mediumblue,
    pdftitle={\Title},
    pdfauthor={\Author},
    breaklinks=true,
}

% Configure header and footer
% Inspired by LaPreprint: https://github.com/roaldarbol/LaPreprint
\renewcommand{\chaptermark}[1]{\markboth{#1}{}}
\newcommand{\Separator}{\hspace{3pt}|\hspace{3pt}}
\newcommand{\FooterFont}{\footnotesize\color{mediumgray}}
\pagestyle{fancy}
\fancyhf{}
\lfoot{%
  \FooterFont{}
  \leftmark{}
}
\rfoot{%
  \FooterFont{}
  \thepage\space de\space \pageref*{LastPage}
}
\renewcommand{\headrulewidth}{0pt}
\renewcommand{\footrulewidth}{1pt}
\preto{\footrule}{\color{lightgray}}
\fancypagestyle{plain}{%
  \fancyhf{}
  \cfoot{%
    \FooterFont{}
    \Title{}
    \Separator{}
    \Author{}
  }
}

% Define fancy text boxes
\NewEnviron{summarybox}[1]{%
  \mdfdefinestyle{summarybox_}{%
    leftline=true,
    rightline=false,
    topline=false,
    bottomline=false,
    linewidth=3pt,
    linecolor=mediumblue,
    backgroundcolor=lightblue,
    innertopmargin=12pt,
    innerbottommargin=12pt,
    innerleftmargin=12pt,
    innerrightmargin=12pt,
    skipbelow=15pt,
    skipabove=15pt,
    frametitleaboveskip=12pt,
    frametitlebelowskip=5pt,
  }
  \newmdenv[style=summarybox_]{summarybox_}
  \begin{summarybox_}[frametitle=#1]
    \BODY
  \end{summarybox_}
}

% Make a list with no margin and smaller spacing for use with the summaryboxes
\NewEnviron{listnomargin}[1]{%
  % Remove spacing between enumerate/itemize items
  \setlist{nosep}
  \begin{#1}[leftmargin=*]
    \BODY
  \end{#1}
}

%%%%%%%%%%%%%%%%%%%%%%%%%%%%%%%%%%%%%%%%%%%%%%%%%%%%%%%%%%%%%%%%%%%%%%%%%%%%%%%
\begin{document}

\pagestyle{empty}
\frontmatter

\begin{titlepage}
  \begin{center}
    \includegraphics[height=1.5cm]{figures/usp.png}
    \hfill
    \includegraphics[height=1.5cm]{figures/iag.png}
    \vspace{9cm}

    \textbf{\Huge \MakeUppercase{\Title{}}}
    \vspace{2cm}

    \textbf{\LARGE \Author{}}
    \vfill

    Departamento de Geofísica
    \\
    Instituto de Astronomia, Geofísica e Ciências Atmosféricas
    \\
    Universidade de São Paulo
    \vspace{2cm}

    \Date{}
  \end{center}
\end{titlepage}

\tableofcontents

\mainmatter
\pagestyle{fancy}

%==============================================================================
\chapter{Introdução}

Este documento relata meus planos para o período entre setembro de 2025
e agosto de 2026. As atividades estão dividades entre ensino de graduação,
atuação na pós-graduação, cultura e extensão, inclusão e pertencimento,
atividades administrativas e pesquisa, inovação e orientações.

\chapter{Ensino em Graduação}

Continuarei ministrando as disciplinas de graduação:

\begin{enumerate}
    \item AGG0011 - Problemas Integrados em Ciências da Terra I
    \item AGG0110 - Elementos de Geofísica
    \item AGG0669 - Gravimetria e Magnetometria Aplicadas à Prospecção de Bens Minerais e Estruturas Crustais
\end{enumerate}

Para 2026, pretendo reestruturar a disciplina AGG0669 junto com os Profs. Eder
Molina e Renata Constantino. Nosso objetivo é reduzir o número de créditos
desta disciplina obrigatória de 6 para 4. O motivo da redução é diminuir o peso
desta disciplina que é cursada no oitavo período da graduação e possibilitar
a criação de uma disciplina optativa que complementará a formação em métodos
potenciais dos alunos interessados.
A criação desta disciplina optativa possibilitará abordar mais temas de
fronteira na área do que somos capazes de fazer no tempo limitado da atual
AGG0669. Nosso plano é elaborar a ementa da nova disciplina e realizar as
mudanças em AGG0669 no começo de 2026 para as mudanças sejam implementadas em
2027 se aprovadas pela CG.

Em 2025, me foi concedida uma bolsa PUB na vertente de Ensino para
criação de um software (chamado Tremelique; \url{https://www.fatiando.org/tremelique}) e recursos educacionais para o ensino de conceitos
básicos de sismologia.
O software permite realizar simulações da propagação de ondas elásticas no
interior da Terra de forma interativa. Assim, alunos e instrutores podem criar
experimentos para visualizar a física por trás dos métodos sísmicos da
geofísica e da propagação das ondas sísmicas estudadas pela sismologia.
O bolsista Paulo Eduardo Crystal foi selecionado
e iniciará suas atividades em setembro de 2025. Como parte do projeto,
utilizaremos o material desenvolvido na disciplina obrigatória ``AGG0230
- Introdução às Ondas Sísmicas''  do Prof. George Sand para avaliar o impacto
e utilidade do material desenvolvido.

\chapter{Atuação em Pós-Graduação}

Na pós-graduação,
darei continuidade à minha atuação na Comissão de Qualificações,
organizando os exames de qualificação de doutorado e seminários de mestrado do
programa de pós-graduação em geofísica do IAG.
Também continuarei ministrando as disciplinas:

\begin{enumerate}
\item AGG5740 - Teoria de Inversão em Geofísica
\item AGG5957 - Gravimetria e Magnetometria
\item AGG5949 - Tópicos Gerais de Geofísica
\end{enumerate}


\chapter{Cultura e Extensão}

Em fevereiro de 2026, irei ministrar uma nova iteração do curso de extensão
``Kit de sobrevivência digital para cientistas'' na Escola de Verão da
Geofísica. O curso foi ministrado na edição de 2025 da Escola e foi um grande
sucesso. Tivemos mais inscritos do que vagas disponíveis e decidi repetir
o curso em 2026 por conta da alta demanda dos estudantes. O curso contou com
estudantes de graduação e pós-graduação de diversas instituições do Brasil
e também com profissionais de fora das universidades. O material do curso é de
acesso aberto e está disponível em \url{https://github.com/compgeolab/kit}.

Também pretendo cadastrar uma AEX para a criação de experimentos e maquetes
para divulgação de conceitos da estrutura da Terra e do campo geomagnético.
A atividade será liderada pela aluna do curso de Geofísica, Samira Lisboa
Santos que propôs a ideia inicial de construir um modelo da Terra para uso em
atividades em escolas de ensino fundamental e médio. Iremos redigir um projeto
para submeter para as chamadas realizadas pela USP e teremos como objetivo
construir um modelo que não represente as camadas internas da Terra mas que
seja capaz de produzir um campo magnético parecido com o da Terra.



\chapter{Inclusão e Pertencimento}

Pretendo continuar minha atuação no programa de tutoria do curso de Geofísica
e expandir com novas vagas em 2026. A tutoria tem sido muito gratificante
e creio que contribuirá para redução na evasão e um melhor aproveitamento
acadêmico dos estudantes envolvidos.

\chapter{Atividades administrativas}

Continuarei com meus cargos administrativos em 2025 e 2026. Meu mandato como
Presidente da Comissão de Cooperação Nacional e Internacional se encerrará em
março de 2026 e pretendo pedir a recondução, caso seja permitida.

Como representante do IAG no Banco Nacional de Dados Gravimétricos,
participarei da reunião que ocorrerá no Congresso Brasileiro de Geofísica de
2025. Meu principal objetivo como representante é aprimorar o acesso aos dados,
que atualmente requerem o preenchimento de um pedido formal e não possuem uma
licença clara de uso.

\chapter{Pesquisa, Inovação e Orientações}

No próximo ano, pretendo continuar dando ênfase nas seguintes áreas:

\begin{enumerate}
\item Microscopia magnética: Meu grupo tem sido pioneiro na aplicação da
   microscopia para uso em paleomagnetismo, com publicações importantes nos
   últimos dois anos. Nesta linha, pretendo investigar no próximo ano o efeito
   de erros no posicionamento de amostras nos resultados e continuar
   o desenvolvimento do software Magali (https://www.fatiando.org/magali) que
   implementa as metodologias que temos desenvolvido.
\item Dados aerogeofísicos para estudo do continente Antártico: Nesta linha,
   aplicamos e estendemos o método das fontes equivalentes para processar
   e integrar grandes conjuntos de dados aeromagnéticos e aerogravimétricos,
   com ênfase nos dados da Antártica. Nesta linha, pretendo aprimorar
   a metodologia para levar em consideração a curvatura da Terra e testar
   diferentes métodos para a integração de dados geofísicos terrestres, aéreos
   e de satélite.
\end{enumerate}

No primeiro semestre de 2026, quatro de meus orientados e coorientandos
planejam defender suas teses e dissertações:

\begin{enumerate}
\item Gelson F. Souza-Junior: Está finalizando o terceiro artigo que comporá sua
   tese de Doutorado e pretende submetê-la até julho de 2026.
\item Arthur S. Macêdo: Deve submeter sua dissertação de Mestrado até fevereiro de
   2026. Arthur já deu início à redação de um artigo para publicação que
   será utilizado como sua dissertação.
\item Yago M. Castro: Deve submeter sua dissertação de Mestrado até fevereiro de
   2026. Yago está nas etapas finais de produção de resultados e dará início
         à redação de um artigo em seguida para utilizá-lo como sua
         dissertação.
\item Eros K. C. Pereira (coorientando): Deve submeter sua dissertação de Mestrado
   até o final de fevereiro de 2026. Eros já está nas etapas finais de redação
   de um artigo e possui diversas apresentações de seu trabalho em congressos
   nacionais.
\end{enumerate}

Com essas defesas, abrirei novas vagas em meu grupo de pesquisa. Pretendo
recrutar estudantes nas seguintes modalidades:

\begin{enumerate}
\item Mestrado em microscopia magnética. Já possuo um candidato interessado nesta
   vaga que se inscreverá no programa de Mestrado em Geofísica para entrada em
   2026.
\item Mestrado em integração de dados aerogeofísicos utilizando fontes
   equivalentes com aplicação na Antártica. Um possível projeto seria
   um estudo sistemático da aplicação de validação cruzado a problemas de
   fontes equivalentes.
\item Doutorado em integração de dados aerogeofísicos. Os alunos Arthur M.
   Siqueira e Eros K. C. Pereira pretendem ingressar no Doutorado em Geofísica
   do IAG e liderariam projetos na integração de grandes volumes de dados
   magnéticos e de gravidade, respectivamente. Seu principal objetivo é gerar
   novos produtos de dados integrados e aprimorar a sua interpretação
   geológica.
\end{enumerate}

Na iniciação científica, os seguintes projetos estão em andamento e continuarão
em 2026:

\begin{enumerate}
\item Gabriel Aparecido das Chagas Silva: Está concluindo uma bolsa PIBIC e dará
   continuidade com uma bolsa PUB na linha de interpretação de dados de
   gravidade no continente Antártico.
\item Felipe Nascimento Hong: Está pleiteando uma bolsa PIBIC para trabalhar com
   desenvolvimento de um método rápido para correção de terreno de dados de
   gravidade em coordenadas esféricas. Este projeto está inserido na temática
   de dados aerogeofísicos da Antártica e será uma importante ferramenta na
   interpretação destes dados.
\end{enumerate}

Atualmente, possuo duas publicações em revisão, ambas com decisão de ``major
revision'' e que devem ser publicadas em 2025 e 2026. As publicações são
referentes aos preprints:

\begin{enumerate}
\item Souza-Junior et al. (2025). Robust directional analysis of magnetic microscopy images using non-linear inversion and iterative Euler deconvolution. EarthArXiv. doi:10.31223/X5N42F
\item Uppal et al. (2025). Transforming Total Field Anomaly into Anomalous Magnetic Field: Using Dual-Layer Gradient-Boosted Equivalent Sources. EarthArXiv. doi:10.31223/X58B1Q
\end{enumerate}

Pretendo também pleitear recursos da FAPESP para financiar as atividades do
laboratório. Ainda em 2025, elaborarei um projeto para a linha de fomento
``Jovem Pesquisador'' da FAPESP envolvendo ambas as minhas linhas de pesquisa
atuais. Caso não seja bem sucedido, o projeto poderá ser quebrado em partes
menores e submetido como Auxílio à Pesquisa Regular também à FAPESP.


\end{document}
