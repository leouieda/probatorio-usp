% Project de atuação no estágio probatório - 2023-2025
%
%%%%%%%%%%%%%%%%%%%%%%%%%%%%%%%%%%%%%%%%%%%%%%%%%%%%%%%%%%%%%%%%%%%%%%%%%%%%%%%
% Set a class and general configuration
\documentclass[12pt,a4paper,oneside]{book}

%%%%%%%%%%%%%%%%%%%%%%%%%%%%%%%%%%%%%%%%%%%%%%%%%%%%%%%%%%%%%%%%%%%%%%%%%%%%%%%
% Set variables with the title, authors, etc.
\newcommand{\Title}{Projeto de Estágio Docente Final}
\newcommand{\Year}{2025}
\newcommand{\Date}{Agosto de \Year{}}
\newcommand{\Author}{Leonardo Uieda}

%%%%%%%%%%%%%%%%%%%%%%%%%%%%%%%%%%%%%%%%%%%%%%%%%%%%%%%%%%%%%%%%%%%%%%%%%%%%%%%
% Import the required packages
\usepackage[utf8]{inputenc}
\usepackage[TU]{fontenc}
\usepackage[brazil]{babel}
\usepackage{amsmath}
\usepackage{amssymb}
\usepackage{graphicx}
\usepackage{hyperref}
\usepackage{fancyhdr}
\usepackage{geometry}
\usepackage{booktabs}
\usepackage{microtype}
\usepackage{siunitx}
\usepackage{xcolor}
% Improved urls with proper hyphenation
\usepackage{xurl}
% Tweak the look of captions
\usepackage{caption}
% To control the style of section titles
\usepackage{titlesec}
% Import natbib and doi packages
\usepackage[round,authoryear,sort]{natbib}
% Add the bibliography to the table of contents
\usepackage[nottoc,chapter]{tocbibind}
% Reference sections by name
\usepackage{nameref}
% Better handling of footnotes inside summary boxes
\usepackage{footmisc}
% Show dois as links on references
\usepackage{doi}
% Remove extra space between references
\usepackage{natbibspacing}
% Use a different font
\usepackage[scaled=0.9,sfdefault]{notomath}
% Icons and fonts (requires using xelatex or luatex)
\usepackage{fontawesome5}
\usepackage{academicons}
% Control the font size
\usepackage{anyfontsize}
\usepackage{setspace}
% To get the number of pages in the document
\usepackage{lastpage}
\usepackage{ragged2e}
% Control over enumerate and itemize
\usepackage{enumitem}
% To define custom environments
\usepackage{environ}
\usepackage{mdframed}
% To control hyphenation for individual blocks of text
\usepackage{hyphenat}
\usepackage{lipsum}

%%%%%%%%%%%%%%%%%%%%%%%%%%%%%%%%%%%%%%%%%%%%%%%%%%%%%%%%%%%%%%%%%%%%%%%%%%%%%%%
% Configuration of the document

\geometry{%
  left=25mm,
  right=25mm,
  top=20mm,
  bottom=15mm,
  headsep=0mm,
  headheight=0mm,
  footskip=5mm,
  includehead=true,
  includefoot=true
}

% Control line and table row spacing
\onehalfspacing
\renewcommand{\arraystretch}{1.5}

% Make urls use the same font as every other text
\urlstyle{same}

% Set the spacing between bibliography entries (requires natbib)
\setlength{\bibsep}{0pt}

% Prevent footnotes from being broken into multiple pages
\interfootnotelinepenalty=10000

% Customize how Chapter headings are displayed
\titleclass{\chapter}{straight}
\titleformat{\chapter}[display]{\normalfont}{}{0pt}{\onehalfspacing\ifnum\thechapter>0 \Large\thechapter. \fi\huge}[\titlerule]
\titlespacing*{\chapter}{0pt}{10pt}{20pt}

% Custom colors
\definecolor{darkgray}{gray}{0.4}
\definecolor{mediumgray}{gray}{0.5}
\definecolor{lightgray}{gray}{0.9}
\definecolor{mediumblue}{HTML}{2060c2}
\definecolor{lightblue}{HTML}{f7faff}

% Configure captions
\captionsetup[table]{position=below,skip=0pt}
\captionsetup{labelfont=bf,font={small,color=darkgray},skip=10pt}

% Configure hyperref and add PDF metadata
\hypersetup{
    colorlinks,
    allcolors=mediumblue,
    pdftitle={\Title},
    pdfauthor={\Author},
    breaklinks=true,
}

% Configure header and footer
% Inspired by LaPreprint: https://github.com/roaldarbol/LaPreprint
\renewcommand{\chaptermark}[1]{\markboth{#1}{}}
\newcommand{\Separator}{\hspace{3pt}|\hspace{3pt}}
\newcommand{\FooterFont}{\footnotesize\color{mediumgray}}
\pagestyle{fancy}
\fancyhf{}
\lfoot{%
  \FooterFont{}
  \leftmark{}
}
\rfoot{%
  \FooterFont{}
  \thepage\space de\space \pageref*{LastPage}
}
\renewcommand{\headrulewidth}{0pt}
\renewcommand{\footrulewidth}{1pt}
\preto{\footrule}{\color{lightgray}}
\fancypagestyle{plain}{%
  \fancyhf{}
  \cfoot{%
    \FooterFont{}
    \Title{}
    \Separator{}
    \Author{}
  }
}

% Define fancy text boxes
\NewEnviron{summarybox}[1]{%
  \mdfdefinestyle{summarybox_}{%
    leftline=true,
    rightline=false,
    topline=false,
    bottomline=false,
    linewidth=3pt,
    linecolor=mediumblue,
    backgroundcolor=lightblue,
    innertopmargin=12pt,
    innerbottommargin=12pt,
    innerleftmargin=12pt,
    innerrightmargin=12pt,
    skipbelow=15pt,
    skipabove=15pt,
    frametitleaboveskip=12pt,
    frametitlebelowskip=5pt,
  }
  \newmdenv[style=summarybox_]{summarybox_}
  \begin{summarybox_}[frametitle=#1]
    \BODY
  \end{summarybox_}
}

% Make a list with no margin and smaller spacing for use with the summaryboxes
\NewEnviron{listnomargin}[1]{%
  % Remove spacing between enumerate/itemize items
  \setlist{nosep}
  \begin{#1}[leftmargin=*]
    \BODY
  \end{#1}
}

%%%%%%%%%%%%%%%%%%%%%%%%%%%%%%%%%%%%%%%%%%%%%%%%%%%%%%%%%%%%%%%%%%%%%%%%%%%%%%%
\begin{document}

\pagestyle{empty}
\frontmatter

\begin{titlepage}
  \begin{center}
    \includegraphics[height=1.5cm]{figures/usp.png}
    \hfill
    \includegraphics[height=1.5cm]{figures/iag.png}
    \vspace{9cm}

    \textbf{\Huge \MakeUppercase{\Title{}}}
    \vspace{2cm}

    \textbf{\LARGE \Author{}}
    \vfill

    Departamento de Geofísica
    \\
    Instituto de Astronomia, Geofísica e Ciências Atmosféricas
    \\
    Universidade de São Paulo
    \vspace{2cm}

    \Date{}
  \end{center}
\end{titlepage}

\tableofcontents

\mainmatter
\pagestyle{fancy}

%==============================================================================
\chapter{Ensino em Graduação}

Continuar ministrando:

AGG0011 - Problemas Integrados em Ciências da Terra I

AGG0110 - Elementos de Geofísica

AGG0669 - Gravimetria e Magnetometria Aplicadas à Prospecção de Bens Minerais e Estruturas Crustais

Reestruturar AGG0669 e criar uma disciplina optativa de métodos potenciais.

Também propus em meu projeto o desenvolvimento de recursos educacionais abertos. Em 2025, me foi concedida uma bolsa PUB na vertente de Ensino para criação de um software e recursos educacionais para o ensino de conceitos básicos de sismologia. O bolsista Paulo Eduardo Crystal foi selecionado e iniciará suas atividades em setembro de 2025. Como parte do projeto, utilizaremos o material desenvolvido na disciplina obrigatória AGG0230 - Introdução às Ondas Sísmicas do Prof. George Sand para avaliar o impacto e utilidade do material desenvolvido.

\chapter{Atuação em Pós-Graduação}

Continuar

AGG5740 - Teoria de Inversão em Geofísica

AGG5957 - Gravimetria e Magnetometria

AGG5949 - Tópicos Gerais de Geofísica

Continuar com as qualificações.


\chapter{Cultura e Extensão}

Ministrar o curso do kit em 2026.

Cadastrar AEX de criação de experimentos e maquetes. Pedir auxílio e PUB para Samira Lisboa Santos.



\chapter{Inclusão e Pertencimento}

Continuar a tutoria e expandir em 2026 para mais alunos.

Continuar com confraternizações do lab.

\chapter{Atividades administrativas}

Participar da reunião do BNDG.

Continuar como Presidente CCNI. Tentar recondução quando mandato acabar em
março de 2026.



\chapter{Pesquisa e Inovação e Orientações}

Orientações para concluir.

Orientações iniciando.

Publicações em andamento.

Colaboração da India com o Jorg.

Projeto FAPESP.

%==============================================================================
\backmatter
\bibliographystyle{apalike-doi}
\bibliography{references}
\chaptermark{Referências Bibliográficas}

\end{document}
